
\def\ft{Conceptos}

\begin{frame}[fragile]
  \frametitle{\ft}
  \begin{block}{Los 3 conceptos fundamentales}
    \begin{itemize}
    \item Escape Analysis
    \item Stack Allocation
    \item Scalar Replacement
    \end{itemize}
  \end{block}
\end{frame}

\begin{frame}[fragile]
  \frametitle{\ft}
  \begin{block}{Ejemplo: hash heap}
    \begin{lstlisting}[language=java]
long hash(byte[] d) {
  var h = new CRC32();
  h.update(d);
  return h.getValue();
}
    \end{lstlisting}
  \end{block}
  \vskip0.5cm
  \begin{tikzpicture}[scale=0.8]
    \onslide<2->{\filldraw[draw=black,fill=green!20] (0,3) rectangle +(13,1);}
    \onslide<3->{\filldraw[draw=black,fill=green!50] (0,3) rectangle +(5,1);}
    \onslide<2->{\draw (0,3.5) node[anchor=west] {Stack};}

    \onslide<4-9>{
    \filldraw[draw=black,fill=green!50] (5,3) rectangle +(3,1);
    \draw[draw=black!50]
    (5,3)
    ++(1,0.2) -- ++(0,0.6)
    ++(1,-0.6) -- ++(0,0.6);
    \draw
    (5,3)
    ++(0.5,0.5) node {?}
    ++(1,0) node (d) {$d$}
    ++(1,0) node (h) {$h$};
    }

    \onslide<5->{\filldraw[draw=black,fill=blue!20] (0,0) rectangle +(15,1);}
    \onslide<6->{
      \filldraw[draw=black,fill=blue!50] (1.5,0) rectangle +(0.5,1);
      \filldraw[draw=black,fill=blue!50] (3,0) rectangle +(2,1);
      \filldraw[draw=black,fill=blue!50] (6,0) rectangle +(1,1);
      \filldraw[draw=black,fill=blue!50] (8.5,0) rectangle +(1.5,1);
    }
    \onslide<8>{
      \filldraw[draw=black,fill=blue!50] (11,0) rectangle +(0.5,1);
    }
    \onslide<9->{
      \filldraw[draw=black,fill=blue!50] (2.7,0) rectangle +(2.3,1);
      \fill[pattern={north east lines}] (2.7,0) rectangle +(0.3,1);
      \filldraw[draw=black,fill=blue!50] (10.7,0) rectangle +(0.8,1);
      \fill[pattern={north east lines}] (10.7,0) rectangle +(0.3,1);
    }
    \onslide<10>{
      \filldraw[draw=black,fill=red] (10.7,0) rectangle +(0.8,1);
      \fill[pattern={north east lines}] (10.7,0) rectangle +(0.3,1);
    }
    \onslide<5->{\draw (0,0.5) node[anchor=west] {Heap};}

    \onslide<7-8>{\draw[ptr] (d.south) to[out=270,in=90] (3.1,1.05);}
    \onslide<9>{\draw[ptr] (d.south) to[out=270,in=90] (2.8,1.05);}
    \onslide<7-10>{\draw (4,0.5) node {$\bar d$};}
    \onslide<8>{\draw[ptr] (h.south) to[out=270,in=90] (11.1,1.05);}
    \onslide<9>{\draw[ptr] (h.south) to[out=270,in=90] (10.8,1.05);}
    \onslide<8-10>{\draw (11.25,0.5) node {$\bar h$};}
  \end{tikzpicture}
\end{frame}

\begin{frame}[fragile]
  \frametitle{\ft}
  \begin{block}{Escape Analysis}
    \begin{itemize}
    \item Si un objeto solo vive mientras está activa la llamada al método,
      entonces se puede alojar en la pila
    \item Para confirmar que no sobrevive:
      \begin{itemize}
      \item No se devuelve,
      \item No se guarda en otro objeto
      \item Ni en una variable global
      \item ...
      \end{itemize}
      Esto es {\bfseries Escape Analysis {\it Pa'Trás}}
    \end{itemize}
  \end{block}
\end{frame}

\begin{frame}[fragile]
  \frametitle{\ft}
  \begin{block}{Ejemplo: hash stack}
    \begin{lstlisting}[language=java]
long hash(byte[] d) {
  var h = new CRC32();
  h.update(d);
  return h.getValue();
}
    \end{lstlisting}
  \end{block}
  \vskip0cm
  \begin{tikzpicture}[scale=0.8]
    \onslide<1->{\filldraw[draw=black,fill=green!20] (0,3) rectangle +(13,1);}
    \onslide<1->{\filldraw[draw=black,fill=green!50] (0,3) rectangle +(5,1);}
    \onslide<1->{\draw (0,3.5) node[anchor=west] {Stack};}

    \onslide<1->{
    \filldraw[draw=black,fill=green!50] (5,3) rectangle +(3,1);
    \draw[draw=black!50]
    (5,3)
    ++(1,0.2) -- ++(0,0.6)
    ++(1,-0.6) -- ++(0,0.6);
    \draw
    (5,3)
    ++(0.5,0.5) node {?}
    ++(1,0) node (d) {$d$}
    ++(1,0) node (h) {$h$};
    }

    \onslide<1->{\filldraw[draw=black,fill=blue!20] (0,0) rectangle +(15,1);}
    \onslide<1->{
      \filldraw[draw=black,fill=blue!50] (1.5,0) rectangle +(0.5,1);
      \filldraw[draw=black,fill=blue!50] (6,0) rectangle +(1,1);
      \filldraw[draw=black,fill=blue!50] (8.5,0) rectangle +(1.5,1);
    }
    \onslide<1->{
      \filldraw[draw=black,fill=blue!50] (2.7,0) rectangle +(2.3,1);
      \fill[pattern={north east lines}] (2.7,0) rectangle +(0.3,1);
    }
    \onslide<1>{
      \filldraw[draw=black,fill=blue!50] (10.7,0) rectangle +(0.8,1);
      \fill[pattern={north east lines}] (10.7,0) rectangle +(0.3,1);
    }
    \onslide<1->{\draw (0,0.5) node[anchor=west] {Heap};}

    \onslide<1->{\draw[ptr] (d.south) to[out=270,in=90] (2.8,1.05);}
    \onslide<1->{\draw (4,0.5) node {$\bar d$};}
    \onslide<1>{\draw[ptr] (h.south) to[out=270,in=90] (10.8,1.05);}
    \onslide<1>{\draw (11.25,0.5) node {$\bar h$};}

    \onslide<2->{
      \filldraw[draw=black,fill=green!50] (8,3) rectangle +(0.8,1);
      \fill[pattern={north east lines}] (8,3) rectangle +(0.3,1);
      \draw (8.55,3.5) node {$\bar h$};
      \draw[ptr] (h.south) .. controls (7.5,2) and (8.0,2) .. (8.1,2.95);
    }
    \onslide<3->{
      \draw[decorate,decoration={brace,raise=1mm}]
      (5,4) -- (8.8,4) node[midway,above,yshift=2mm] {checksum};

      \filldraw[draw=black,fill=green!50] (8.8,3) rectangle +(3,1);
      \draw[draw=black!50]
      (8.8,3)
      ++(1,0.2) -- ++(0,0.6)
      ++(1,-0.6) -- ++(0,0.6);
      \draw
      (8.8,3)
      ++(0.5,0.5) node {?}
      ++(1,0) node (uthis) {$this$}
      ++(1,0) node (ud) {$d$};
      \draw[decorate,decoration={brace,raise=1mm}]
      (8.8,4) -- (11.8,4) node[midway,above,yshift=2mm] {update};
      \draw[ptr] (uthis) to[out=270,in=270] (8.1,2.95);
      \draw[ptr] (ud) to[out=270,in=90] (2.8,1.05);
    }
  \end{tikzpicture}
\end{frame}

\begin{frame}[fragile]
  \frametitle{\ft}
  \begin{block}{Implicaciones de colocar objetos completos en la pila}
    \begin{itemize}
    \item Afecta a detalles importantes en GC:
      \begin{itemize}
      \item Write barriers no tendría efectos sobre los objetos en la pila
      \item Busqueda de raíces en la pila
      \item No habría que hacer marcado ni barrido de los objetos en la pila
      \end{itemize}
    \item Puede verse afectado por desoptimizaciones
    \item Mejora el rendimiento
      \begin{itemize}
      \item Es más sencillo colocar el objeto en la pila
      \item No se genera basura
      \item Hay efectos indirectos: más localidad en los datos
      \end{itemize}
    \item Permite sincronizaciones
    \item Permite {\it escape {\bfseries Pa'Lante}}
    \end{itemize}
  \end{block}
\end{frame}


\begin{frame}[fragile]
  \frametitle{\ft}
  \begin{block}{Monstruo 1: Inlining: la madre de todas las optimizaciones}
    Reemplazar una llamada a un método por su código
  \end{block}
\end{frame}

\begin{frame}[fragile]
  %\frametitle{\ft}
  \begin{block}{Código relevante de CRC32}
    \begin{lstlisting}[language=java]
public interface Checksum {
  default void update(byte[] b) {update(b,0,b.length);}
}
public class CRC32 implements Checksum {
  private int crc = 0;

  public long getValue() {return (long)crc & 0xffffffffL;}

  public void update(byte[] b, int off, int len) {
    crc = updateBytes(crc, b, off, len);
  }
  static int updateBytes(int crc,
        byte[] b, int off, int len) {
    return updateBytes0(crc, b, off, len);
  }
  @IntrinsicCandidate
  static native int updateBytes0(int crc,
      byte[] b, int off, int len);
}
    \end{lstlisting}
  \end{block}
\end{frame}

\begin{frame}[fragile]
  \frametitle{\ft}
  \begin{block}{Ejecución reiterada del método {\tt hash}}
\begin{lstlisting}
-XX:+UnlockDiagnosticVMOptions
-XX:+PrintCompilation
-XX:+PrintInlining
\end{lstlisting}
\small
\begin{lstlisting}
::hash (22 bytes)   made not entrant
@ 4   CRC32::<init> (5 bytes)   inline (hot)
  @ 1   Object::<init> (1 bytes)   inline (hot)
@ 10   Checksum::update (11 bytes)   inline (hot)
  @ 5   CRC32::update (38 bytes)   inline (hot)
    @ 31   CRC32::updateBytes (14 bytes)   inline (hot)
      @ 10   CRC32::updateBytes0 (0 bytes)   (intrinsic)
@ 16   CRC32::getValue (10 bytes)   inline (hot)
\end{lstlisting}
  \end{block}
\end{frame}

\begin{frame}[fragile]
  \frametitle{\ft}
  \begin{block}{Método {\tt hash} tras hacer inlining}
    \begin{lstlisting}[language=java]
long hash(byte[] d) {
  var h = new Object() {int crc;};
  h.crc = 0;
  h.crc = updateBytes0(h.crc, d, 0, d.length);
  return (long)h.crc & 0xffffffffL;
}
    \end{lstlisting}
  \end{block}
  \vskip1cm
  \begin{tikzpicture}[scale=0.8]
    \filldraw[draw=black,fill=green!20] (0,3) rectangle +(13,1);
    \filldraw[draw=black,fill=green!50] (0,3) rectangle +(5,1);
    \draw (0,3.5) node[anchor=west] {Stack};

    \filldraw[draw=black,fill=green!50] (5,3) rectangle +(3,1);
    \draw[draw=black!50]
    (5,3)
    ++(1,0.2) -- ++(0,0.6)
    ++(1,-0.6) -- ++(0,0.6);
    \draw
    (5,3)
    ++(0.5,0.5) node {?}
    ++(1,0) node (d) {$d$}
    ++(1,0) node (h) {$h$};

    \filldraw[draw=black,fill=blue!20] (0,0) rectangle +(15,1);
    \draw (0,0.5) node[anchor=west] {Heap};
    \filldraw[draw=black,fill=blue!50] (1.5,0) rectangle +(0.5,1);
    \filldraw[draw=black,fill=blue!50] (6,0) rectangle +(1,1);
    \filldraw[draw=black,fill=blue!50] (8.5,0) rectangle +(1.5,1);

    \filldraw[draw=black,fill=blue!50] (2.7,0) rectangle +(2.3,1);
    \fill[pattern={north east lines}] (2.7,0) rectangle +(0.3,1);
    \draw (4,0.5) node {$\bar d$};
    \draw[ptr] (d.south) to[out=270,in=90] (2.8,1.05);

    \filldraw[draw=black,fill=blue!50] (10.7,0) rectangle +(1,1);
    \fill[pattern={north east lines}] (10.7,0) rectangle +(0.3,1);
    \draw (11.35,0.5) node {crc};
    \draw[ptr] (h.south) to[out=270,in=90] (10.8,1.05);
  \end{tikzpicture}
\end{frame}


\begin{frame}[fragile]
  \frametitle{\ft}
  \begin{block}{Método {\tt hash} tras hacer Scalar Replacement}
    \begin{lstlisting}[language=java]
long hash(byte[] d) {
  int crc = 0;
  crc = updateBytes0(crc, d, 0, d.length);
  return (long)crc & 0xffffffffL;
}
    \end{lstlisting}
  \end{block}
  \vskip1cm
  \begin{tikzpicture}[scale=0.8]
    \filldraw[draw=black,fill=green!20] (0,3) rectangle +(13,1);
    \filldraw[draw=black,fill=green!50] (0,3) rectangle +(5,1);
    \draw (0,3.5) node[anchor=west] {Stack};

    \filldraw[draw=black,fill=green!50] (5,3) rectangle +(3,1);
    \draw[draw=black!50]
    (5,3)
    ++(1,0.2) -- ++(0,0.6)
    ++(1,-0.6) -- ++(0,0.6);
    \draw
    (5,3)
    ++(0.5,0.5) node {?}
    ++(1,0) node (d) {$d$}
    ++(1,0) node (h) {crc};

    \filldraw[draw=black,fill=blue!20] (0,0) rectangle +(15,1);
    \draw (0,0.5) node[anchor=west] {Heap};
    \filldraw[draw=black,fill=blue!50] (1.5,0) rectangle +(0.5,1);
    \filldraw[draw=black,fill=blue!50] (6,0) rectangle +(1,1);
    \filldraw[draw=black,fill=blue!50] (8.5,0) rectangle +(1.5,1);

    \filldraw[draw=black,fill=blue!50] (2.7,0) rectangle +(2.3,1);
    \fill[pattern={north east lines}] (2.7,0) rectangle +(0.3,1);
    \draw (4,0.5) node {$\bar d$};
    \draw[ptr] (d.south) to[out=270,in=90] (2.8,1.05);
  \end{tikzpicture}
\end{frame}


\begin{frame}[fragile]
  \frametitle{\ft}
  \begin{block}{Definiciones}
    \begin{itemize}
    \item Escape Analysis: Determina el ambito dinámico de un objeto
    \item Partial Escape Analysis: Variante más sofisticada que es capaz de
      operar sobre bifurcaciones en el código
    \item Stack Allocation: Coloca objetos {\it completos} en la pila.
      Sólo se puede hacer si no sobrevive al método
    \item Scalar Replacement: Coloca los campos de un objeto en la pila.
      Sólo se puede hacer si no escapa del método (en ningún sentido).
      Los campos no tienen por qué estar consecutivos, y podrían limitarse
      a registros del procesador.
      Es un {\it caso especial} de Stack Allocation pero no una {\it mejora}
    \end{itemize}
  \end{block}
\end{frame}


\begin{frame}[fragile]
  \frametitle{\ft}
  \begin{block}{Estado del arte}
    \begin{description}
    \item[C2]
      Escape Analysis,
      Scalar Replacement para objetos
    \item[GraalVM]
      Partial Escape Analysis,
      Scalar Replacement para objetos,
      Scalar replacement limitado para arrays
    \item[OpenJ9]
      Partial Escape Analysis,
      {\bf Stack Allocation} para objetos,
      Scalar Replacement para objetos
    \item[Azul Zing]
      Partial Escape Analysis,
      Scalar Replacement para objetos,
      Scalar replacement limitado para arrays
    \end{description}
  \end{block}
\end{frame}
