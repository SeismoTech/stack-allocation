% En el título:
% - Voy a hablar de Stack Allocation y otros elementos de la java, más bien de
% la jvm, que son relevantes, que tienen una interacción, unas veces positiva
% y otras no tanto, pero veremos cómo soslayar los problemas

\section{Toma de contacto}
\def\ft{Toma de contacto}

% - Antes de empezar con la chicha, 2 puntualizaciones

% - Hay algo de información, no mucha, sobre SA en la red.
% Por una parte, hay información muy técnica, relativa a algoritmos de EA o
% a implementaciones en la JVM.
% Por otra, hay información muy superficial de la operativa de SA o muy
% puntual sobre algún detalle.
% El planteamiento de esta charla es distinto. Nace de la experiencia de
% intentar aprovechar SA en programas o librerías reales para conseguir
% una mejora en el código, simultáneamente en rendimiento y en legibilidad.

% - Y ya que hablamos de rendimiento, una definición somera de SA podría ser
% esta ...
% Por ahora vamos a destacar la palabra *optimización*.
% Si una charla habla de mejoras de rendimiento en la JVM, eso mola,
% porque la gente que trabaja en esas cosas tiene derecho a preocuparse
% por el rendimiento de nuestros programas.
% Pero si una charla habla de cómo mejorar el rendimiento de nuestros
% propios programas, entonces entra en una zona pantanosa,
% y necesita una pequeña introducción expiatoria
% porque en otro caso le arrojarán la famosa frase
% ...

\begin{frame}[fragile]
  \frametitle{\ft}
  \begin{block}{Definición}
    Stack Allocation es una
    \only<1>{optimización}\only<2>{{\bf\color{red} optimización}}
    por la que un objeto creado con {\tt new~C(...)}
    se aloja realmente en la pila
  \end{block}
\end{frame}

\begin{frame}[fragile]
  \frametitle{\ft}
  \begin{block}{Donald Knuth / Tony Hoare}
    \visible<2>{We should forget about small efficiencies, say about 97\% of the time:}
    {\only<2>{\color{gray}}Premature optimization is the root of all evil.}
    \visible<2>{Yet we should not pass up our opportunities in that critical 3\%}
  \end{block}
\end{frame}

\begin{frame}[fragile]
  \frametitle{\ft}
  \begin{block}{Diseñar para}
    \begin{itemize}
    \item Corrección
    \item Mantenimiento
    \item Seguridad
    \item Testeo
    \item $\cdots$
    \visible<2>{\item {\bf Rendimiento}}
    \end{itemize}
  \end{block}
\end{frame}
